\documentclass[12pt]{article}

\usepackage{cancel}
\usepackage{graphicx} 
\usepackage{amssymb}
\usepackage[portuguese]{babel}
\usepackage[a4paper,top=3cm,bottom=2cm,left=3cm,right=2cm]{geometry}
\usepackage{ascii}
\usepackage{amsmath}
\usepackage{amssymb}
\usepackage{amsfonts} 
\usepackage{tikz}
\usepackage[utf8]{inputenc}
\usepackage{listings}
\usepackage{parskip}
\linespread{1.5}
\usepackage{enumitem}
\title{Aula do dia 08 de dezembro de 2023}
\author{Lucas Amaral Taylor}
\date{\today}

\begin{document}
    \maketitle
    \section*{Definição de continuidade}
    $X \subseteq \mathbb{R}$ e $f: X \to \mathbb{R}$. Dado $x_0 \in X$, dizemos que $f$ \textbf{é contínua em $x_0$} quando:
    \begin{equation*}
        \forall \varepsilon > 0 \text{, } \exists \delta = \delta(x_0, \varepsilon) > 0 \text{ tal que } \left( x \in X \text{ e } \left| x - x_0 \right| < \delta \implies \left| f(x) - f(x_0)\right| < \varepsilon \right)
    \end{equation*}
    
    \subsection*{Exemplo}
    $X = [0,1) \cup (2,2]$
    \begin{equation*}
        f(x)=\left\{    
        \begin{array}{l}
            x, \text{ se } 0 \leq x < 1 \\
            x+1, \text{ se } 1 <x \leq 2
        \end{array}
        \right.
    \end{equation*}
    \subsubsection*{Exercício}
    Mostre que $f$ é contínua usando $\varepsilon$ e $\delta$.
    
    \subsection*{Exemplo}
    $f: \mathbb{R} \to \mathbb{R}$
    
    $x \longmapsto f(x) = x^2$ é contínua em todo ponto de $\mathbb{R}$. 
    
    \subsubsection*{Prova}
    Fixado $x_0 \in \mathbb{R}$. Para cada $\varepsilon > 0$, queremos mostrar que: 
    \begin{equation*}
        |f(x) - f(x_0)| = |x^2 - x_0^2| = |x + x_0| \cdot |x - x_0| < \varepsilon
    \end{equation*}
    \textbf{Quando $x \in X = \mathbb{R}$ e $|x-x_0| < \delta$  para algum $\delta > 0$ adequado.}
    
    Por exemplo: $\delta = \frac{\varepsilon}{2}$ será que funciona? Temos:
    \begin{equation*}
        |f(x) - f(x_0)| = |x + x_0| \cdot |x - x_0| \cdot \frac{\varepsilon}{2} < \varepsilon \text{ (?) } 
    \end{equation*}
    
    \textit{Uma pequeno lembrete que é necessário levar em conta:}
    \begin{equation*}
        |x| - |y| \leq ||x|-|y|| \leq |x+y|
    \end{equation*}
    \begin{equation*}
        |x| - |y| \leq ||x| - |y|| \leq |x-y|
    \end{equation*}
    
    \textbf{Para qualquer ponto de $\mathbb{R}$}
    
    \textit{Continuando o exemplo apresentado...}
    
    \begin{equation*}
        ||x| - |x_0|| \leq |x+x_0| 
    \end{equation*}
    
    Sem perda de generalidade, vamos tomar  $0 < \delta < 1$. Daí:
    \begin{equation*}
        |x| - |x_0| \leq |x - x_0| < \delta < 1
    \end{equation*}
    \begin{equation*}
        \implies |x| - |x_0| < 1 \implies |x| < 1 + |x_0|
    \end{equation*}
    
    Ou seja, 
    \begin{equation}
        |x| < 1 + |x_0| \text{ quando } |x-x_0| < \delta \text{ para } (0 < \delta < 1)
        \label{eq:rel-imp}
    \end{equation}
    
    Observe que:
    \begin{equation*}
        |f(x) - f(x_0)| = |x + x_0| \cdot |x - x_0|
    \end{equation*}
    
    Pela desigualdade triangular, vem:
    \begin{equation*}
        (|x| + |x_0|) \cdot |x-x_0| 
    \end{equation*}
    
    Por \ref{eq:rel-imp}, temos que:
    \begin{equation*}
         (|x| + |x_0|) \cdot |x-x_0| \leq (1+ |x_0| + |x_0|) \cdot |x- x_0|
    \end{equation*}

    \subsubsection*{Conclusão}
    Dado $\varepsilon > 0$, 
    \begin{equation*}
        |f(x) - f(x_0)| \leq (1+2|x_0|)|x-x_0| < \cancel{(1 + 2|x_0|)} \frac{\varepsilon}{\cancel{1+ 2|x_0|}} = \varepsilon
    \end{equation*}
    
    Quando $0 < \delta < 1$ e $|x - x_0| < \delta$

    \textbf{Tomamos}
    \begin{equation*}
        \delta (\varepsilon, x_0) = \delta = \min\left\{\frac{\varepsilon}{(1 + 2|x_0|)}, \frac{1}{2} \right\} > 0
    \end{equation*}
    
    \textbf{Provamos que: } Dado $\varepsilon > 0$ e $x_0 \in \mathbb{R}$. \textbf{Tomando $\delta = \min\left\{\frac{\varepsilon}{(1+2|x_0|)}, \frac{1}{2} \right\} > 0$ vale que $|f(x) - f(x_0)| = |x^2 - x_0^2| < \varepsilon$}. Ou seja, $f(x) = x^2$ ]e contínua em $x_0$.

    \subsubsection*{Exercício (muito importante)}
    \begin{equation*}
        f:\left[0, +\infty \right] \longmapsto \mathbb{R}
    \end{equation*}
    \begin{equation*}
        x \longmapsto f(x) = \sqrt{x}
    \end{equation*}

    Mostre que $f$ é contínua usando $\varepsilon$ e $\delta$.

    \textbf{Dica:} Trate separadamente $x_0 = 0$. e $x_0 \neq 0$.

    \subsection*{Exemplo} 
    \begin{equation*}
        f: \mathbb{R} \to \mathbb{R}
    \end{equation*}
    \begin{equation*}
        f(x) = 5x
    \end{equation*}

    \subsubsection*{Prova}
    Dados $\varepsilon > 0$ e $x_0 \in \mathbb{R}$
    \begin{equation*}
        |f(x) - f(x_0)| = |5x - 5x_0| = 5 \cdot |x - x_0| < \cancel{5}\frac{\varepsilon}{\cancel{5}} = \varepsilon
    \end{equation*}

    Tome $\delta = \frac{\varepsilon}{5} > 0$.

    \subsubsection*{Exercício}
    Mostre que dado $a \in \mathbb{R}$ a função $f: \mathbb{R} \to \mathbb{R}$, $f(x) = a \cdot x$ em contínua em dado ponto $x_0 \in \mathbb{R}$ \textbf{(usando $\varepsilon \text{ e } \delta$). Mostre que SEMPRE podemos escolher $\delta$ dependendo apenas de $\varepsilon$ e não de $x_0$}.

    \newpage
    \section*{Definição de ponto isolado}
    Seja $X \subseteq \mathbb{R}$. Dizemos que $x_0 \in X$ é um \textbf{ponto isolado de $X$} quando existe $\varepsilon_{x_0} > 0$ tal que $B_{\varepsilon_{x_0}} \cap X = \left\{x_0\right\}$

    \subsection*{Exercício}
    Seja $X \subseteq \mathbb{R}$ tal que todos os pontos de $X$ são isolados. Mostre que $X$ é enumerável.

    \subsection*{Exercício}
    Seja $f: X \to \mathbb{R}$ uma função. $X \subseteq \mathbb{R}$ e $x_0 \in X$ é isolado. Mostre que $f$ é contínua em $x_0$.

    \subsection{Exemplo}
    \begin{equation*}
        f: \mathbb{R} \to \mathbb{R}
    \end{equation*}

    \begin{equation*}
        x \longmapsto f(x) = \chi_{\mathbb{Q}}(x) = \left\{ 
        \begin{array}{l}
            1, \text{ se } x \in \mathbb{Q} < \\
            0, \text{ se } x \notin \mathbb{Q}
        \end{array}
        \right.
    \end{equation*}

    \subsubsection*{Prova}
    Seja $x_0 \in \mathbb{Q}$. A mostrar: $\exists \varepsilon > 0$ tal que $\forall \delta >0$. $|f(x) - f(x_0)| > \varepsilon$

    Por fim, temos que:
    \begin{equation*}
        \forall \varepsilon>0, \exists \delta>0 \text { tal que }
\end{equation*}
\begin{equation*}
    x \in X \text{ e }|x-x|<\delta \Rightarrow|f(x)-f(x_0)|<\varepsilon 
\end{equation*}
\begin{equation*}
    \exists \varepsilon>0, \forall \delta>0
\end{equation*}
Vale que existe $x_\delta \in X$ tal que 
\begin{equation*}
    \left|x_\delta-x_0\right|<\delta \text { e }\left|f\left(x_\delta\right)-f\left(x_0\right)\right| \geqslant \varepsilon
\end{equation*}

\end{document}